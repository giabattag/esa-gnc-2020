\documentclass{esagnc}

% Abstract submission deadline: 27 February
% => TASF deadline: skipped
% ==> First draft deadline: ~10 February
% Paper submission deadline: 31 July
% => TASF deadline: ~15 July
% ==> First draft deadline: ~22 June

% Max: 25 words
\title{Development and optimization of a robust vision-based navigation algorithm for space-representative hardware}
\author[(1)]{Giacomo Battaglia}
\author[(2)]{Anthea Comellini}
\author[(3)]{Francesco Capolupo}
\author[(1)]{Paolo Panicucci}
\affil[(1)]{Department of Aerospace Science and Technology, Politecnico di Milano, Via Giuseppe la Masa, 34, 20156 Milan, \{giacomo.battaglia, paolo.panicucci\}@polimi.it}
\affil[(2)]{Thales Alenia Space, Cannes, 06150, France, anthea.comellini@thalesaleniaspace.com}
\affil[(3)]{European Space Agency, Noordwijk, 2200 AG, Netherlands, francesco.capolupo@esa.int}

\footauth{G. Battaglia, A. Comellini, F. Capolupo, P. Panicucci}

\begin{document}

\maketitle

% NOTE: the template is not updated, got from Github

% Max: 500 words
% 467 words
\begin{abstract}
Close proximity operations are a critical phase of In-Orbit Servicing (IOS) missions, requiring the GNC system to ensure safety while meeting stringent performance requirements.
When approaching a non-cooperative target, the chaser must rely solely on its sensors to determine the relative pose (i.e. position and attitude).
Furthermore, the target might not be fully known onboard, meaning that there may be some uncertainty in the modeling of its visual appearance (e.g. surface optical properties, moving appendages configurations, etc.) and its dynamical properties (i.e. center of mass, inertia).
Vision-based navigation (VBN) is one of the possible techniques that enables non-cooperative pose estimation. Monocular cameras are generally baselined as relative navigation sensors because of their low size, weight, and power. The VBN pipeline comprises two functional modules: the image processing, which extracts contextual information from the input images to generate a relative pose (or analogous) measurement, and the navigation filter, which refines the pose estimate by leveraging on a propagation model for the relative dynamics.
In VBN algorithm design, hardware is a major driver. Indeed, space hardware imposes tight constraints on memory, computational power, and latency, which particularly affect image processing tasks due to the inherent data-intensive nature of images. Therefore, on-board hardware selection imposes a series of trade-offs that have a strong influence on the overall performance of the navigation chain. This article presents the optimization and deployment process of a robust AI-aided VBN algorithm on representative space hardware. The proposed end-to-end navigation solution, which includes image processing and a relative navigation filter, is tailored to be robust to uncertainties in the mass properties of the target. This is achieved on the AI-based image processing side by training a keypoint regression network on a dedicated synthetic dataset, rendered with the high-fidelity SpiCam renderer developed by Thales Alenia Space and featuring a target with variable shape. In the navigation filter, uncertainties are modeled within a Schmidt–Kalman Multiplicative Extended Kalman Filter by augmenting the state with center-of-mass and inertia parameters. The algorithm is tailored and optimized for space hardware. This is achieved by a preliminary feasibility assessment of the functional blocks for hardware selection, followed by hardware-specific optimizations.
The navigation filter was tested on the GR740 board (LEON4 single core), proving to satisfy both a 10 Hz propagation requirement and a 1 Hz update requirement, leaving approximately 0.7 s for neural network execution.
The neural network was implemented on CPU, GPU, and FPGA platforms. The GR740 CPU exhibited a runtime greater than 25 s, while the ZedBoard (ARM Cortex-A9) achieved execution times below 4 s, compatible with a 0.2 Hz update rate. The GPU and FPGA implementations, instead, met the 1 Hz requirement. However, the algorithm’s numerical performance is re-evaluated after quantization and graph-level optimizations, which generally impact output accuracy. This article presents the results of these design and development activities through processor-in-the-loop testing, comparing multiple representative hardware platforms.
\end{abstract}

% \section*{Acknowledgments}

% This work was conducted under EISI Agreement No. 4000144147, within the Open Space Innovation Platform (OSIP), through the support of the European Space Agency (ESA) and Thales Alenia Space France (TAS-F).

% \bibliographystyle{esacit}
% \bibliography{references}

\end{document}
